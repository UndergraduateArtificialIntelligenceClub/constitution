\documentclass[11pt]{article}
\usepackage{geometry} % See geometry.pdf to learn the layout options. There are lots.
\geometry{letterpaper} % ... or a4paper or a5paper or ...
%\geometry{landscape} % Activate for for rotated page geometry
\usepackage[parfill]{parskip} % Activate to begin paragraphs with an empty line rather than an indent
\usepackage{titling}
\date{\today}

\usepackage{graphicx}
\usepackage{amssymb,latexsym} %maths
\usepackage{amsmath,amsfonts,amsthm} %maths
\usepackage{xfrac} %special fractions
\usepackage{paralist} %special listing environments
\usepackage{color} %color

\title{Constitution \\
of the \\
University of Alberta \\
Undergraduate Artificial Intelligence Society \\
}

\date{Amended and Ratified \thedate}

\author{Revision I}

\begin{document}
\maketitle
%\raggedright

\begin{enumerate}[I.]
  \item Name
    \begin{enumerate}[A)]
      \item The group shall be known as the University of Alberta Undergraduate Artificial Intelligence Society, hereinafter referred to as ``the UAIS" or ``the society". The society exists under the student groups section of the University of Alberta Students' Union Constitution (at least 75\% Undergraduate Students).
      \item The symbol of the Undergraduate Artificial Intelligence Society shall be a coloured neural representation of the brain graph.
      \end{enumerate}
  %%%%%%%%%
  \item Mandate

    The society has a responsibility to provide the following to its membership:
    \begin{enumerate}[A)]
      \item Supporting a diverse community of students interested in learning AI.
      \item Teaching the basic tools to program an artificial intelligence system including neural networks, and applying them to projects.
      \begin{enumerate}[a)]
      \item There should be at least two ``workshops" per semester where a new skill is introduced.
      \item These projects will be team collaborative efforts, with students learning how to work on a team with leadership roles and using version control systems.
      \item A project is defined as any participation to the organization to a club sanctioned event, participate to a workshop, or contribute code on our GitHub repositories
      \end{enumerate} 
      \item Representing the interest of members before external organizations such as the Students' Union and AMII.
      \item Maintaining the meeting place in the Student Innovation Centre.
    \end{enumerate}
  %%%%%%%%%
  \item Membership: As there are many diverse areas in AI, it's important to respect unique perspectives. No matter if you're a beginner or expert, there’s a place for you.
    \begin{enumerate}[A)]
      \item Levels of Membership and Eligibility
        \begin{enumerate}[i)]
          \item Associative Membership
            \begin{enumerate}[a)]
              \item Staff, academic or non-academic, may obtain associate membership.
            \end{enumerate}
          \item Full Membership
          \begin{enumerate}[a)]
              \item Any undergraduate student or graduate student, full or part-time, may become a full member of the UAIS during the academic year in which they are registered at the University of Alberta. 
              \item The terms ``full member'' and ``voting member'' are
                considered equivalent and may be used interchangeably.
              \item A full member is committed to work on at least one AI related project and attend at least two meetings per academic year.
            \end{enumerate}
        \end{enumerate}
      \item Privileges
        \begin{enumerate}[i)]
          \item Associative Members
            \begin{enumerate}[a)]
              \item Each associate member is entitled to the use of the meeting place for extracurricular projects and use of its resources.
              \item Each associate member is entitled to attend all general meetings and Society events, albeit without voting privileges or other preferential status.
            \end{enumerate}
          \item Full Members
            \begin{enumerate}[a)]
              \item Each full member is entitled to attend all general meetings and Society events.
              \item Each full member may vote at a general meeting. These votes may have significant influence for the club, such as allocation of funds, or on executive positions.
            \end{enumerate}
        \end{enumerate}
      \item Terms of Membership
        \begin{enumerate}[i)]
          \item Membership lasts from the beginning of an academic year to the end of the academic year as defined in the UofA calendar.
        \end{enumerate}
      \item Suspension
        \begin{enumerate}[i)]
          \item Any member may have their membership suspended under Article III, Section D during a general meeting by a majority vote. Valid grounds to suspend membership are a violation of the University of Alberta Code of Student Behaviour or the UAIS constitution.
        \end{enumerate}
      \item Reinstatement
        \begin{enumerate}[i)]
          \item Any member suspended may have their membership reinstated during a general meeting by a two-thirds vote.
        \end{enumerate}
    \end{enumerate}
  %%%%%%%%%
  \item Meetings
    \begin{enumerate}[A)]
      \item General Meeting

      A general meeting may be called by any full member and shall be
        considered valid if and only if all of the following conditions are
        satisfied:
        \begin{enumerate}[i)]
          \item 40\% of the non-executive voting membership must be in
            attendance.
          \item Three members of the executive must be in attendance, one of
            whom must be either the president or the vice-president.
          \item If the general secretary is not present, a designate will be appointed to take the minutes.
          \item The general meeting must have been advertised at least five
            business days prior to its occurrence.
        \end{enumerate}
      \item Initial General Meeting

      The optional initial general meeting of the UAIS in each academic year shall be held no more than three weeks after the commencement of the fall session. This meeting shall be extensively advertised no less than one week prior to its occurrence.

      \item Annual General Meeting

      In March, a special general meeting shall be called for the purpose of
        electing the executive officers. More than 50\% of the total voting
        membership must attend this meeting for the results to be considered
        binding. The newly elected executive will take office at the beginning of the next academic year. The current executive will make themselves available for operational training until that time.

      \item Executive Meetings

      In order for an executive meeting to be considered valid, the following
        conditions must be satisfied.
        \begin{enumerate}[i)]
          \item Quorum must be established; at least three of the executive
            members, including either the president or vice-president, must attend.
          \item  If the general secretary is not present, a designate will be appointed to take the minutes.
        \end{enumerate}
      \item Committee Meetings

      Committee meetings shall be scheduled and conducted at the discretion of
        the committee chair.
      \item Emergency Impeachment Meetings

      Should any full member wish to impeach an executive member, the following criteria must be met.
        \begin{enumerate}[i)]
          \item At least two-thirds of the voting membership must meet.
          \item An emergency chair must be nominated and elected by a majority
            vote of those present.
          \item The emergency chair must appoint a secretary to record the
            discussion and outcome of the impeachment vote as per Article V,
            Section C.
        \end{enumerate}
      \item Minutes
        \begin{enumerate}[i)]
          \item Full minutes for every UAIS meeting, both general and executive,
            must be taken and made available to the membership at large in both
            electronic and printed form no later than one week following the
            meeting.
          \item These minutes must be approved by the next meeting of the body
            which produced them.
        \end{enumerate}
      \item Chairmanship

      Every general and executive meeting will be chaired by the president or,
      in their absence, by the vice-president. In the event of an emergency
      impeachment meeting, this section shall be superseded by Article IV,
      Section F, Subsection (ii). Should any conflict arise as to the procedure
      used during a meeting, Robert's Rules of Order shall be considered as the
      definitive source.
      \item Motions

      All motions presented to a general meeting shall be considered passed by a majority vote in favour of the motion except as specified by certain
      sections of this document.
    \end{enumerate}
  %%%%%%%%%
  \item Society Executive
  Graduate students may only run for the position of Vice President. All other roles must be held by undergraduate students. The total number of Vice President positions may be composed by no more than 50\% graduate students.
    \begin{enumerate}[A)]
      \item Elected Officers (in order of precedence)
        \begin{enumerate}[i)]
          \item President

          It is the responsibility of the President to
            \begin{enumerate}[a)]
              \item Chair all meetings of the UAIS when present.
              \item Oversee the responsibilities of executive members in booking presentations, planning events, and working with communications to ensure the fulfillment of the Society mandate.
              \item Oversee, in cooperation with the treasurer, the financial affairs of the UAIS.
              \item Serve as liaison with external groups and individuals.
              \item Represent the interests of the UAIS before any external
                organization.
              \item Attend mandatory Student Group Services President training.

            \end{enumerate}
          \item Vice President

          It is the responsibility of the Vice President to
            \begin{enumerate}[a)]
              \item Assist the president in all of their responsibilities as
                delegated by the president.
              \item Assume all duties of the president in the event of their
                absence.
             \item \textit{Member Learning:} Provide moral support and assistance to members as necessary.
              \item Plan and organize social presentations with the assistance of the Presentation Committee, if formed, established under Article VI, Section D, Subsection iii below.
             \item Assist the VP External with the social media accounts, if necessary.
            \end{enumerate}
          \item Treasurer

          It is the responsibility of the Treasurer to
            \begin{enumerate}[a)]
              \item Maintain detailed records of all financial transaction undertaken by the UAIS.
              \item Report the financial status of the Society to the membership at all general meetings.
              \item Oversee the collection of all generated revenues.
              \item Administrate all disbursements of Society funds.
              \item Plan, organize, and oversee the execution of fundraising activities with the assistance of the Fundraising Committee, if formed, established under Article VI, Section D, Subsection ii.
             \item Attend mandatory Student Group Services Treasurer training.
              \item Make monthly deposits of funds obtained by the Society.
            \end{enumerate}
        \item General Secretary
        
           It is the responsibility of the General Secretary to 
           \begin{enumerate}[a)]
              \item Take minutes at all meetings at which they are present and ensure that these minutes are available to the membership as Article IV, Section G.
              \item  Keep membership records in order.
              \item Keep an active list of all ongoing projects the club is engaged in.
            \end{enumerate}
          \item VP External

          It is the responsibility of the VP External to
            \begin{enumerate}[a)]
              \item Provide moral support and assistance to members as necessary.
              \item Act as liaison between the Society and external organizations, alongside the president.
               \item Plan and organize social presentations with the assistance of the Presentation Committee, if formed, established under Article VI, Section D, Subsection iii below.
          	\item Oversee social media roles, in conjunction with the vice-president.
            \end{enumerate}
            
            \item VP Internal
            
            It is the responsibility of the VP Internal to
            \begin{enumerate}[a)]
            \item Provide moral support and assistance to members as necessary.
            \item Assist the executive in the preparation and delivery of Society communications.
              Optionally edit, publish, and distribute the society
                newsletter with the assistance of the Newsletter Committee, if
                formed, established under Article VI, Section D, Subsection i.
            \end{enumerate}
            
            \item VP Technology
            
            It is the responsibility of the VP Technology to
            \begin{enumerate}[a)]
            \item Handle password management.
            \item Manage the GitHub organization.
            \item Manage the Google Drive, both public and private.
            \item Manage the Google Suite account (uais@ualberta.ca).
            \item Handle deployments on any cloud providers such as Amazon Web Services.
            \item Maintain the website.
            \item Be in charge of any other technology orientated tasks defined by the President.
            \end{enumerate}
        
            \item VP Media
                
            It is the responsibility of the VP Media to
            \begin{enumerate}[a)]
            \item Run social media (i.e Instagram, LinkedIn, Google calendar etc).
            \item Advertise events (digital posters etc).
            \item Create Eventbrite events.
            \item Improve branding.
            \item Design, plan, and execute effective marketing campaigns.
            \end{enumerate}
        \end{enumerate}
        \item Succession

        In the event that any member of the elected executive resigns their
        position or is impeached as per section C below, a temporary
        replacement shall be appointed by the executive until a by-election
        can be held. This election shall be held at the next general meeting
        following the resignation or impeachment. More than 50\% of the total membership must be in attendance in order for the results to be
        considered binding.
        \item Impeachment

        Any member of the executive may be impeached in either of the
        following two ways.
          \begin{enumerate}[i)]
            \item Four of the eight executive members vote to impeach in
              either an executive or general meeting.
            \item A full member calls an emergency impeachment meeting as per Article IV, Section F.
          \end{enumerate}
        \item Term of Office

        The term of office of each executive officer shall extend from the beginning of the academic year (May 1st)
        until the end of the academic year (April 30th) as per Article IV, Section C.

        \item Past Executive Advisor

        Immediately upon their election, the President shall appoint a
        non-returning member of the previous year's executive to serve in an
        advisory role. In the case of a full returning executive, no advisor
        will be appointed. It is understood that all outgoing executive will
        make their contact information available to incoming executive.
    \end{enumerate}
  %%%%%%%%%
  \item Committees
    \begin{enumerate}[A)]
      \item A committee shall consist of an odd number of voting members,
        chaired by an executive officer. The executive officer shall abstain
        from voting except in the event it is required to break a tie.
      \item Committees are to be formed by the UAIS executive for a specific task and are responsible only to the executive.
      \item The executive may disband a committee at any time.
      \item Optional Standing Committees

      The following optional standing committees shall meet on a regular basis,
        if formed, during the academic year. The makeup of these committees
        shall be determined by the executive at the Initial General Meeting. At
        their discretion, the executive may remove a member from a standing
        committee, provided that they are replaced at the earliest possible
        opportunity.
        \begin{enumerate}[i)]
          \item Newsletter Committee
            \begin{enumerate}[a)]
              \item The VP internal shall chair the committee.
              \item The committee shall be responsible for editing, publishing,
                and distributing the society newsletter.
            \end{enumerate}
          \item Fundraising Committee
            \begin{enumerate}[a)]
              \item The Treasurer shall chair the committee.
              \item The committee shall be responsible for the planning,
                coordination, and execution of fundraising activities.
            \end{enumerate}
          \item Presentation Committee
            \begin{enumerate}[a)]
              \item The VP-external shall chair the committee.
              \item The committee shall be responsible for the planning and
                organization of presentations (there must be at least two per semester).
            \end{enumerate}
        \end{enumerate}
    \end{enumerate}
  %%%%%%%%%
  \item Elections
    \begin{enumerate}[A)]
      \item Any member is eligible for office if and only if they are a full
        member in good standing. No member may hold more than one elected
        position. All members of the executive must be current undergraduate
        students.
      \item Elections for all executive positions shall be held at the Annual
        General Meeting as defined in Article IV, Section C. Bi-elections are
        held as required in accordance with Article V, Section B.
      \item  Nominations for each executive position will close one day prior
        to the voting for that position. Nominations by either motion or
        volunteer shall be accepted. Nominations must be accepted either in
        person or by signed letter.
        \begin{enumerate}[i)]
          \item Executive positions shall be voted for independently in order
            of precedence as per Article V, Section A.
          \item A deputy returning officer (``DRO'') who is not standing for
            any position shall be appointed by the meeting chair and the final
            decision on all voting will be the right of the DRO.
          \item A scrutineer for each candidate is allowed, provided the
            scrutineer is not the candidate themself.
          \item Voting shall be by secret ballot. Voting by proxy is not
            permitted.
          \item Absentee balloting is allowed under the circumstance that a
            member can not be present at the meeting. Absentee ballots must be
            submitted in a sealed envelope, signed across the seal, after the
            complete list of nominees has been released and prior to the
            beginning of voting.
          \item Results will be validated by a majority vote. In the case that
            no candidate achieves a majority, the candidate who received the
            least votes, determined by random lot administered by the DRO if
            there is a tie, shall be stricken from the ballot. The position
            shall be voted for again. This procedure shall continue until a
            majority is reached or, if a tie occurs with only two candidates
            remaining, the winner shall be determined by a random lot
            administered by the DRO.
        \end{enumerate}
    \end{enumerate}
  %%%%%%%%%
  \item Financial Administration
    \begin{enumerate}[A)]
      \item Disbursement of Funds
        \begin{enumerate}[i)]
          \item Any disbursement of funds greater than five hundred dollars
            (\$500) must be approved by motion in general meeting.
          \item Any disbursement of funds of five hundred dollars (\$500) or
            less shall be at the discretion of any executive member subject to
            the passing of a subsequent motion at an executive meeting.
          \item Signing authority on any UAIS accounts shall be restricted to
            exactly the Treasurer and President.
          \item Any member may be reimbursed money put towards the operation of the Society as approved by the executive.
        \end{enumerate}
      \item Record Keeping and Budget
        \begin{enumerate}[i)]
          \item A detailed ledger of financial affairs shall be maintained by the Treasurer and made available to the general membership as per article V, Section A, Subsection iii, Paragraphs a and b.
          \item A budget for the school year shall be created and approved by the executive team.
          \item The budget shall be updated by the Treasurer prior to
            executive meetings throughout the year.
          \item The budget shall be posted online and made available to the
            general membership as per article V, Section A, Subsection iii,
            Paragraphs a and b.
        \end{enumerate}
      \item Audits
        \begin{enumerate}[i)]
          \item The executive shall audit the financial records of the UAIS for one financial year beginning the end of an academic year and ending the following year. This audit shall occur between the executive elections in March and May 1, under the supervision of the incoming executive.
          \item Any member may audit the financial records by requesting a
            meeting with the treasurer.
        \end{enumerate}
    \end{enumerate}
  %%%%%%%%%
  \item Amendments and Ratification

    Amendments to this document made by a full member must be submitted in
    writing to the executive and will be immediately tabled for discussion and vote at the next general meeting. A two-thirds majority vote is necessary to approve a constitutional amendment. The constitution will be reviewed once every two years and ratified if needed. The general secretary will maintain documentation of review periods when the constitution is reviewed but not ratified.
  \item Disbandment
    \begin{enumerate}[A)]
      \item The Undergraduate Artificial Intelligence Society may be disbanded by a unanimous vote of the entire voting membership at a general meeting.
      \item  In the event of disbandment, all funds shall revert to the Faculty of Science, University of Alberta.
      \item In this case, UAIS must contact the University (via Student Group Services) to indicate the group has disbanded.
    \end{enumerate}
\end{enumerate}

These amendments ratified \thedate\ by:

President: Mira Patel \\
Vice President: Siddhartha Chitrakar \\
Treasurer: Jacob Winch \\
General Secretary: Frances Igwe \\
VP External: Brian Cheng \\
VP Internal: Taranveer Purewal \\
VP Technology: Aarush Bhat \\
VP Media: Yong Lee \\


\end{document}
